\documentclass[14pt, a4paper]{extarticle}

\usepackage[T2A]{fontenc}
\usepackage[utf8]{inputenc}
\usepackage[russian]{babel}

\pagestyle{empty}

\begin{document}
$E = mc^{2}$
Это формула из общей теории относительности Альберта Эйнштейна.
Показывает эквивалентность массы и энергии.
Формула была выведение еще в начале XX века.
\end{document}
